\documentclass{article}
\usepackage[utf8]{inputenc}
\usepackage{enumitem}

\newcommand{\namesigdate}[2][5cm]{%
  \begin{tabular}{@{}p{#1}@{}}
    #2 \\[2\normalbaselineskip] \hrule \\[0pt]
    {\small \textit{Datum och plats}} \\[2\normalbaselineskip] \hrule \\[0pt]
    {\small \textit{Signatur}}
  \end{tabular}
}

\title{XP-R - Stadgar}
\author{Robert Hedman}
\date{February 2017}

\begin{document}

\maketitle

\newpage

\tableofcontents

\newpage

\section{Grundläggande underlag}


\subsection{Föreningens namn}
Föreningens juridiska namn är XP-R, Experimentell Robotik.

\subsection{Sätesort}

XP-R har sitt säte i Luleå.

\subsection{Syfte}

XP-R syftar att underlätta för sina medlemmar byggandet och/eller utvecklandet av robotar och andra projekt.
XP-R tillhandahåller, i den mån möjligt, en 3D-verkstad för sina medlemmar.
XP-R är en ideell och politiskt obunden förening.

\section{Allmänna regler}

\subsection{Verksamhetens drift}
Verksamheten drivs av styrelsen. Styrelsen underhåller och vid behov utvecklar, i den mån möjligt, 3D-verkstaden. Styrelsen har rätt att upprätta och upprätthålla regler för allmän trivsel i 3D-verkstaden.


\section{Centrala organ}
\subsection{Beslutande organ}
XP-R består utav en styrelse som driver föreningen. Styrelsen är det beslutsfattande organet. Styrelsen bör vid större beslut hålla ett protokollfört möte. Styrelsen är det högsta beslutsfattande organet efter årsmötet.  För att beslut skall vara giltiga vid styrelsemöten måste en majoritet rösta för beslutet. Vid lika röstning för eller mot beslutet är ordförandens röst utslagsröst.

\subsubsection{Styrelsens poster}

Styrelsens poster består minst utav en ordförande, en sekreterare samt en kassör.
Vid behov kan även ett rimligt antal ledamöter och suppleanter tillsättas.
En person kan ha flera poster.
Styrelsen väljs på årsmötet.
Styrelsen ska se till att:
\begin{itemize}[noitemsep]
    \item årsmötets beslut verkställs
    \item stadarna följs
    \item Förbereda nästa års årsmöte och lägga fram verksamhetsberättelse och bokslut samt proposition på verksamhetsplan till årsmötet
    \item Sköta ekonomin för XP-R.
\end{itemize}


\subsubsection{Kallelser till möten}
Kallelser till styrelsemöten behöver ej utfärdas på något speciellt vis. För att ett möte ska vara giltigt måste en majoritet av styrelsen vara närvarande, fysiskt eller via annan länk (till exempel digital länk, telefon, etc). Ingen speciell dagordning måste följas. Mötet måste vara protokollfört för att vara giltigt.

\subsubsection{Firmatecknare}
Vid årsmötet utses firmatecknare. Årsmötet fastställer antal, samt vilka som är, firmatecknare. Regler och villkor för firmatecknarna fastställs också. 

\subsection{Revisor}
Revisorn väljs på årsmötet. Revisorn har till uppgift att granska verksamhetsrapporten och bokslutet. Revisorn skall sedan sammanställa en revisorsberättelse som årsmötet kan användas som grund för att bevilja eller neka den avgående styrelsen ansvarsfrihet. Revisorn bör vara en extern person som inte är medlem i föreningen.

\subsection{Valberedning}
Om årsmötet anser det nödvändigt kan en valberedning tillsättas, annars får styrelsen ta hand om valberedningen på det sätt dom anser lämpligt. 

Medlemmar får aktivt kandidera till poster inför årsmötet fram till och med att posten skall tillsättas.

Om inga kandidater till de tre huvudposterna vid ett årsmöte finns måste årsmötet upplösa föreningen.

Den styrelsen som grundar XP-R har rätt att behålla sina poster i styrelsen i max 3 år utan att bli avsatta om ej brottslig verksamhet förekommer.

\subsection{Verksamhets- och räkneskapsår}
Verksamhetsåret, tillika räkenskapsår, omfattar tiden 1 januari till och med sista december.

\subsection{Ändring av stadgarna}
För ändring av stadgarna krävs 2/3 delars majoritet vid ett årsmöte.

\subsection{Upplösning av föreningen}
Årsmötet kan besluta att upplösa föreningen vid beviljad ansvarsfrihet. Det är den avgående styrelsens ansvar att ev. upplösa föreningen.

\subsubsection{Föredelning av tillgångar}
Vid upplösningen av XP-R får styrelsen besluta om fördelningen av tillgångar. Om ingen fördelning aktivt tas fram övergår ägandet av alla tillgånger till SRT; instutionen för system och rymdteknik vid Luleå Tekniska Universitet.


\section{Medlemskap}
Alla personer som är inskriva vid något program på universitet kan ansöka om medlemskap och vid betald medlemsavgift antas som medlem. Andra personer som ansöker om medlemskap skall granskas av styrelsen och beviljas medlemskap (efter betald medlemsavgift) om styrelsen anser det lämpligt. Det krävs en majoritet av styrelsens röster vid ett möte för att bevilja annan medlem.
\subsection{Medlemsavgift}
Medlemsavgiften per år för en medlem fastställs av årsmötet. Styrelsemedlemmar behöver inte betala avgiften men får om dom vill.

\subsection{Rösträtt}
Alla medlemmar har rösträtt endast vid årsmötet. Alla styrelsemedlemmar har rösträtt vid styrelsemöten.

\subsection{Uteslutning}
Om styrelsen anser att en medlem skadar föreningen kan styrelsen vid ett styrelsemöte rösta om uteslutning av medlem. För uteslutning krävs 2/3-delars majoritet vid styrelsemötet. Utvisning får inte vara på diskriminerande grunder. Vid uteslutningar måste grunderna för utelsutningen vara väl motiverade och dokumenterade. 

En utesluten medlem har rätt till en omprövning vid nästföljande årsmöte. Då måste medlemmen aktivt lämna in en skriftlig ansökan om omprövning till styrelsen minst en arbetsdag innan årsmötet.


\section{Årsmöte}
Årsmötet är XP-R högsta beslutande organ.
Årsmötet skall hållas efter verksamhetsårets slut men inte mer än två månader efter.
\subsection{Kallelse}
För att ett årsmöte skall vara giltigt måste en kallelse till årsmöte skickas ut till samtliga medlemmar minst 1 vecka innan årsmötet. Kallelsen behöver inte innehålla någon annan information än tid och plats.
\subsection{Röstberättigade medlemmar}
Medlemmar som är närvarande vid årsmötet har rösträtt.

\subsection{Godkännande av dagordning}
Vid årsmötet måste dagordningen godkännas. För detta krävs att en majoritet av närvarande röstberättigade medlemmar röstar för dagordningen. Närvarande röstberättigade medlemmar får föreslå nya punkter i dagordningen tills dess att dagordningen godkänns. Om dagordningen ej kan godkönnas med rimliga medel kan revisorn godkänna dagordningen,

\subsection{Val av styrelse}
För att en styrelse skall utses måste en majoritet på årsmötet rösta för den nya styrelsen.

\subsection{Punkter som skall behandlas}
Dessa punkter måste behandlas vid ett årsmöte:
\begin{itemize}[noitemsep]
    \item Årsmötets öppnande
    \item Årsmötets stadgeenliga utlysande.
    \item Fastställande av röstlängd
    \item Val av mötesordförande, mötessekreterade samt två mötesrösträknare tillika protokolljusterare.
    \item Val av ny styrelse (om inte föreningen upplöses)
    \item Val av revisor
    \item Avgående styrelsens verksamhetsberättelse
    \item Avgående styrelsens bokslut samt kassarapport
    \item Revisorsberättelse
    \item Beviljande av den avgående styrelsens ansvarsfrihet
    \item Bestämmande av medlemsavgift
    \item Motioner till årsmötet
    \item Övriga ärenden
    \item Årsmötets avslut
\end{itemize}

\section{Extra årsmöte}
Om styrelsen anser det nödvändigt kan ett extra Årsmöte utlysas. Detta möte skall följa samma regler som ett vanligt årsmöte förutom datumet.


\vspace{1cm}


\noindent \namesigdate[10cm]{Vittne 1: Viktor Kasimir}
\vspace{0.5cm}

\noindent \namesigdate[10cm]{Vittne 2: Johan Carlstedt}
\vspace{0.5cm}

\end{document}
